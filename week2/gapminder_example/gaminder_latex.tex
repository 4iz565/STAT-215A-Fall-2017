\documentclass[english]{article}\usepackage[]{graphicx}\usepackage[]{color}
%% maxwidth is the original width if it is less than linewidth
%% otherwise use linewidth (to make sure the graphics do not exceed the margin)
\makeatletter
\def\maxwidth{ %
  \ifdim\Gin@nat@width>\linewidth
    \linewidth
  \else
    \Gin@nat@width
  \fi
}
\makeatother

\definecolor{fgcolor}{rgb}{0.345, 0.345, 0.345}
\newcommand{\hlnum}[1]{\textcolor[rgb]{0.686,0.059,0.569}{#1}}%
\newcommand{\hlstr}[1]{\textcolor[rgb]{0.192,0.494,0.8}{#1}}%
\newcommand{\hlcom}[1]{\textcolor[rgb]{0.678,0.584,0.686}{\textit{#1}}}%
\newcommand{\hlopt}[1]{\textcolor[rgb]{0,0,0}{#1}}%
\newcommand{\hlstd}[1]{\textcolor[rgb]{0.345,0.345,0.345}{#1}}%
\newcommand{\hlkwa}[1]{\textcolor[rgb]{0.161,0.373,0.58}{\textbf{#1}}}%
\newcommand{\hlkwb}[1]{\textcolor[rgb]{0.69,0.353,0.396}{#1}}%
\newcommand{\hlkwc}[1]{\textcolor[rgb]{0.333,0.667,0.333}{#1}}%
\newcommand{\hlkwd}[1]{\textcolor[rgb]{0.737,0.353,0.396}{\textbf{#1}}}%
\let\hlipl\hlkwb

\usepackage{framed}
\makeatletter
\newenvironment{kframe}{%
 \def\at@end@of@kframe{}%
 \ifinner\ifhmode%
  \def\at@end@of@kframe{\end{minipage}}%
  \begin{minipage}{\columnwidth}%
 \fi\fi%
 \def\FrameCommand##1{\hskip\@totalleftmargin \hskip-\fboxsep
 \colorbox{shadecolor}{##1}\hskip-\fboxsep
     % There is no \\@totalrightmargin, so:
     \hskip-\linewidth \hskip-\@totalleftmargin \hskip\columnwidth}%
 \MakeFramed {\advance\hsize-\width
   \@totalleftmargin\z@ \linewidth\hsize
   \@setminipage}}%
 {\par\unskip\endMakeFramed%
 \at@end@of@kframe}
\makeatother

\definecolor{shadecolor}{rgb}{.97, .97, .97}
\definecolor{messagecolor}{rgb}{0, 0, 0}
\definecolor{warningcolor}{rgb}{1, 0, 1}
\definecolor{errorcolor}{rgb}{1, 0, 0}
\newenvironment{knitrout}{}{} % an empty environment to be redefined in TeX

\usepackage{alltt}

\usepackage{geometry}
\geometry{verbose,tmargin=1in,bmargin=1in,lmargin=1in,rmargin=1in}
\usepackage{fancyhdr}
\pagestyle{fancy}
\setlength{\parskip}{\smallskipamount}
\setlength{\parindent}{0pt}
\usepackage{amsthm}
\usepackage{amsmath}
\usepackage{url}
\usepackage{float} % required for fig.pos = "H"

\title{Gapminder exploration}
\author{Rebecca Barter}
\IfFileExists{upquote.sty}{\usepackage{upquote}}{}
\begin{document}

\maketitle

\section{Introduction}

Gapminder is an excellent organization aimed at increasing the use and understanding of statistics on a number of global topics. They collect a variety of data from many sources and aim to produce fact-based statistics reflecting the current state of our world. The data we are exploring throughout this analysis consists of population, life expectency and GDP information for many countries through time.

The data can be found from \url{https://raw.githubusercontent.com/resbaz/r-novice-gapminder-files/master/data/gapminder-FiveYearData.csv} if you would like to download it yourself.





Fortunately, the data was already very clean, so we did not conduct any major modifications to the data.





\section{Visualizing the gapminder data (ggplot2)}


We are interested in exploring life expectancy as a function of GDP. Figure \ref{fig:gdp-life} shows a scatterplot of life expectancy versus GDP.

\begin{knitrout}
\definecolor{shadecolor}{rgb}{0.969, 0.969, 0.969}\color{fgcolor}\begin{figure}[H]

{\centering \includegraphics[width=\maxwidth]{figure/gdp-life-1} 

}

\caption[Life expectancy versus GDP for all countries in the year 2007]{Life expectancy versus GDP for all countries in the year 2007}\label{fig:gdp-life}
\end{figure}


\end{knitrout}


It certainly appears as though there is some kind of rapid increase in the low GDP range, which slows to a gradual increase in the high GDP range. Several African countries have surprisingly low life expectency for their GDP.

Next, we explore change in life expectancy over time. Figure \ref{fig:life-time} shows a series of boxplots, one for each year-continent combination. Each data point corresponds to the life expectency of a country for the given year in the given continent.




\begin{knitrout}
\definecolor{shadecolor}{rgb}{0.969, 0.969, 0.969}\color{fgcolor}\begin{figure}[H]

{\centering \includegraphics[width=\maxwidth]{figure/life-time-1} 

}

\caption[Life expectancy over time]{Life expectancy over time}\label{fig:life-time}
\end{figure}


\end{knitrout}


We see that the life expectancy increased in Africa from 1950 up until the 1990s but has stayed fairly constant with a median of around 50 years since the 1990s. The Americas, Asia, and Europe on the other hand, have experienced continued growth. 

\subsection{Comparing GDP across continents (dplyr)}

Table 1 compares GDP per capita across continents.




% latex table generated in R 3.3.2 by xtable 1.8-2 package
% Thu Aug 31 15:39:12 2017
\begin{table}[H]
\centering
\begin{tabular}{lrrr}
  \hline
continent & countries & mean & SD \\ 
  \hline
Oceania &   2 & 29810.19 & 6540.99 \\ 
  Europe &  30 & 25054.48 & 11800.34 \\ 
  Asia &  33 & 12473.03 & 14154.94 \\ 
  Americas &  25 & 11003.03 & 9713.21 \\ 
  Africa &  52 & 3089.03 & 3618.16 \\ 
   \hline
\end{tabular}
\caption{A table displaying the mean and standard deviation of GDP per capita in 2007 for each continent} 
\label{gdp_table}
\end{table}



Clearly Oceania is leading the way in terms of GDP per cap! Go Aussies!

Next, we want to ask about raw GDP (i.e. overall GDP for each country, rather than standardized by per capita). Table 2 shows the average total GDP for each continent for 2007.


% latex table generated in R 3.3.2 by xtable 1.8-2 package
% Thu Aug 31 15:39:12 2017
\begin{table}[H]
\centering
\begin{tabular}{lrrr}
  \hline
continent & countries & mean & SD \\ 
  \hline
Americas & 25 & 777 & 2573 \\ 
  Asia & 33 & 628 & 1344 \\ 
  Europe & 30 & 493 & 678 \\ 
  Oceania & 2 & 404 & 424 \\ 
  Africa & 52 & 46 & 92 \\ 
   \hline
\end{tabular}
\caption{A table displaying the mean and standard deviation of GDP (in billions) in 2007 for each continent} 
\label{total_gdp_table}
\end{table}


I didn't get a chance to write up examples about tidyr, but feel free to follow the example at http://swcarpentry.github.io/r-novice-gapminder/14-tidyr/.

\end{document}
